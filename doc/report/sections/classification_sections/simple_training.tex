\documentclass[../../main.tex]{subfiles}
\begin{document}

Unlike regression, which has a simple linear form, doing a full Bayesian analysis is more difficult.

Therefore, the classification model is trained by finding the parameters which maximise the posterior:

\begin{align*}
\theta^*
&= \argmax_{\theta}\left[p(\theta | \mathcal{D}) \right] \\
&= \argmax_{\theta}\left[\ln p(\theta | \mathcal{D}) \right] \\
&= \argmax_{\theta}\left[\ln p({\{y_n\}}_{n=1}^N|{\{\tilde{\vect{x}}_n\}}_{n=1}^N, \theta) + \ln p(\theta) \right] \\
&= \argmax_{\theta}\left[\mathcal{L}(\theta) + \ln p(\theta) \right] \\
&= \argmin_{\theta}\left[\mathcal{C}(\theta)\right]
\end{align*}

$\mathcal{L} = \ln p({\{y_n\}}_{n=1}^N|{\{\tilde{\vect{x}}_n\}}_{n=1}^N, \theta)$ is defined as the log-likelihood of the parameters.
The cost $\mathcal{C}(\theta)$ is the expression we need to minimise in order to optimise the parameters, where:
\[ \mathcal{C}(\theta) = -\mathcal{L}(\theta) - \ln p(\theta)\]

Optimisation is performed as a minimisation problem, simply because this is the conventional method, so standard methods perform minimisation of an expression.

The expression for $\mathcal{C}(\theta)$ cannot be minimised analytically, so instead gradient descent is used.
This only requires that the gradient of the cost with respect to the parameters can be found.

\subsubsection{Finding the log-likelihood}

Some terms which will be used later:
\[ \tilde{X} = \left[\begin{matrix}
    \uparrow & \uparrow & & \uparrow \\
    \tilde{\vect{x}}_1 &
    \tilde{\vect{x}}_2 &
    \cdots &
    \tilde{\vect{x}}_N \\
    \downarrow & \downarrow & & \downarrow
\end{matrix}\right] \]

\[ W = \left[\begin{matrix}
    \uparrow & \uparrow & & \uparrow \\
    \vect{w}_1 &
    \vect{w}_2 &
    \cdots &
    \vect{w}_C \\
    \downarrow & \downarrow & & \downarrow
\end{matrix}\right] \]

\[ \textrm{Discrete delta function  } \delta_c(y) =
\begin{cases}
1 & c = y \\
0 & c \neq y
\end{cases}
\]

\[ \vect{y} = \left[\begin{matrix}
    \delta_1(y) \\
    \delta_2(y) \\
    \vdots \\
    \delta_C(y)
\end{matrix}\right]
\quad
\textrm{ie:}\quad {[\vect{y}]}_c =
\begin{cases}
1 & c = y \\
0 & c \neq y
\end{cases}
\]

\[ Y = \left[\begin{matrix}
    \uparrow & \uparrow & & \uparrow \\
    \vect{y}_1 & \vect{y}_2 & \cdots & \vect{y}_N \\
    \downarrow & \downarrow & & \downarrow
\end{matrix}\right]\]

\[ \vect{p}_n = \left[\begin{matrix}
        p_Y(1|\tilde{\vect{x}}_n) \\
        p_Y(2|\tilde{\vect{x}}_n) \\
        \vdots \\
        p_Y(C|\tilde{\vect{x}}_n)
\end{matrix}\right]\]

\[ P = \left[\begin{matrix}
    \uparrow & \uparrow & & \uparrow \\
    \vect{p}_1 & \vect{p}_2 & \cdots & \vect{p}_N \\
    \downarrow & \downarrow & & \downarrow
\end{matrix}\right]\]

Expanding the log-likelihood:

\begin{align*}
\mathcal{L}(\theta)
&= \ln p({\{y_n\}}_{n=1}^N|{\{\tilde{\vect{x}}_n\}}_{n=1}^N, \theta) \\
&= \ln \prod_{n=1}^N p(y_n | \tilde{\vect{x}}_n, \theta) \\
&= \sum_{n=1}^N \ln p(y_n | \tilde{\vect{x}}_n, \theta)
\end{align*}

Before expanding out the pdf, the following results can be used:
\[ \vect{w}_y^T\tilde{\vect{x}} = {(W\vect{y})}^T\tilde{\vect{x}}\]

Now, expanding the pdf:
\begin{align*}
p(y_n | \tilde{\vect{x}}_n, \theta)
&=
\frac{\exp{\left(\vect{w}_{y_n}^T\tilde{\vect{x}}_n\right)}}{\sum_{c=1}^C \exp{\left(\vect{w}_c^T\tilde{\vect{x}}_n\right)} } \\
&=
\frac{\exp{\left({(W\vect{y}_n)}^T\tilde{\vect{x}}_n\right)}}{\sum_{c=1}^C \exp{\left(\vect{w}_c^T\tilde{\vect{x}}_n\right)} }
\end{align*}

\begin{align*}
\ln p(y_n | \tilde{\vect{x}}_n, \theta)
&=
{(W\vect{y}_n)}^T\tilde{\vect{x}}_n - \ln \sum_{c=1}^C \exp{\left(\vect_c{w}^T\tilde{\vect{x}_n}\right)}
\end{align*}

Finally:
\[ \mathcal{L}(\theta) =
\sum_{n=1}^N \left\{{(W\vect{y}_n)}^T\tilde{\vect{x}}_n - \ln \sum_{c=1}^C \exp{\left(\vect_c{w}^T\tilde{\vect{x}_n}\right)} \right\}
\]

\subsubsection{Finding the prior}

Each weight has the same prior, a multivariate Gaussian with covariance matrix $\sigma_w^2I$.

\begin{align*}
p(\theta) &= \prod_{c=1}^C p(\vect{w}_c) \\
          &\propto \prod_{c=1}^C \exp{\left(-\frac{\vect{w}_c^T\vect{w}_c}{2\sigma_w^2}\right)}
\end{align*}

\begin{align*}
    \ln p(\theta) &= - \sum_{c=1}^C \frac{\vect{w}_c^T\vect{w}_c}{2\sigma_w^2} + \textrm{const}
\end{align*}

And because we only care about optimising over the parameters, constants can be removed from the cost.

\subsubsection{Expression for the cost}

\begin{align*}
\mathcal{C}(\theta)
&=
- \mathcal{L}(\theta) - \ln p(\theta) \\
&=
\sum_{n=1}^N \left\{-{(W\vect{y}_n)}^T\tilde{\vect{x}}_n + \ln \sum_{c=1}^C \exp{\left(\vect{w}_c^T\tilde{\vect{x}_n}\right)} \right\}
+ \sum_{c=1}^C \frac{\vect{w}_c^T\vect{w}_c}{2\sigma_w^2}
\end{align*}

\subsubsection{Gradient of the cost}

To encapsulate the gradient over all parameters, the matrix derivative of the cost can be found, with respect to $W$.
\begin{align*}
\frac{d\mathcal{C}}{dW}
&=
\sum_{n=1}^N \left\{ -\vect{y}_n \tilde{\vect{x}}_n^T
+ \frac
    {\frac{d}{dW}\left(\sum_{c=1}^C \exp{\left(\vect{w}_c^T\tilde{\vect{x}}_n\right)}\right)}
    {\sum_{c=1}^C \exp{\left(\vect{w}_c^T\tilde{\vect{x}}_n\right)}}
\right\}
+ \frac{1}{2\sigma_w^2} \sum_{c=1}^C \frac{d}{dW}(\vect{w}_c^T\vect{w}_c) \\
&=
- \sum_{n=1}^N \left\{ \vect{y}_n \tilde{\vect{x}}_n^T \right\}
+ \sum_{n=1}^N \sum_{c=1}^C \left\{
    \frac{d}{dW}\left[\vect{w}_c^T\tilde{\vect{x}}_n \right]
    \frac
        {\exp{\left(\vect{w}_c^T\tilde{\vect{x}}_n\right)}}
    {\sum_{k=1}^C \exp{\left(\vect{w}_k^T\tilde{\vect{x}}_n\right)}}
\right\}
+ \frac{1}{2\sigma_w^2} \sum_{c=1}^C \frac{d}{dW}[\vect{w}_c^T\vect{w}_c] \\
&=
- Y\tilde{X}^T
+ \sum_{n=1}^N \sum_{c=1}^C \left\{
    \frac{d}{dW}\left[\vect{w}_c^T\tilde{\vect{x}}_n \right]
    p(c|\tilde{\vect{x}}_n)
\right\}
+ \frac{1}{2\sigma_w^2} \sum_{c=1}^C \frac{d}{dW}[\vect{w}_c^T\vect{w}_c] \\
\end{align*}

Tackling the more complex expression

\[
\sum_{c=1}^C \left\{
    \frac{d}{dW}\left[\vect{w}_c^T\tilde{\vect{x}}_n \right]
    p(c|\tilde{\vect{x}}_n)
\right\}
\]
\[
\frac{d(\vect{w}_c^T\tilde{\vect{x}}_n)}{d\vect{w}_i} =
\begin{cases}
    \tilde{\vect{x}}_n^T & c = i \\
    \vect{0}^T & c \neq i
\end{cases}
\]
\[
\frac{d(\vect{w}_c^T\tilde{\vect{x}}_n)}{dW} =
\left[\begin{matrix}
    \vect{0}^T \\
    \vdots \\
    \leftarrow \tilde{\vect{x}}_n^T \rightarrow \\
    \vdots \\
    \vect{0}^T
\end{matrix}\right]
\begin{matrix} \\ \textrm{(c'th row)} \\ \\ \end{matrix}
\]
\[
\sum_{c=1}^C \left\{
    \frac{d}{dW}\left[\vect{w}_c^T\tilde{\vect{x}}_n \right]
    p(c|\tilde{\vect{x}}_n)
\right\} =
\left[\begin{matrix}
    p(1|\tilde{\vect{x}}_n)\tilde{\vect{x}}_n^T \\
    p(2|\tilde{\vect{x}}_n)\tilde{\vect{x}}_n^T \\
    \vdots \\
    p(C|\tilde{\vect{x}}_n)\tilde{\vect{x}}_n^T
\end{matrix}\right] =
\vect{p}_n\tilde{\vect{x}}_n^T
\]
\[
\sum_{n=1}^N \sum_{c=1}^C \left\{ \ldots \right\} =
\sum_{n=1}^N \vect{p}_n\tilde{\vect{x}}_n^T = P\tilde{X}^T
\]

Then, the expression which comes from the prior:
\[ \frac{d(\vect{w}_c^T\vect{w}_c)}{d\vect{w}_i} =
\begin{cases}
    2\vect{w}_c^T & i = c \\
    \vect{0}^T & i \neq c
\end{cases}
\]
\[ \frac{d(\vect{w}_c^T\vect{w}_c)}{dW} =
\left[\begin{matrix}
    \vect{0}^T \\
    \vdots \\
    \leftarrow 2\tilde{\vect{w}}_c^T \rightarrow \\
    \vdots \\
    \vect{0}^T
\end{matrix}\right]
\begin{matrix} \\ \textrm{(c'th row)} \\ \\ \end{matrix}
\]
\[ \frac{1}{2\sigma_w^2}\sum_{c=1}^C \frac{d(\vect{w}_c^T\vect{w}_c)}{dW} = \frac{1}{2\sigma_w^2}
\left[\begin{matrix}
    \leftarrow 2\tilde{\vect{w}}_1^T \rightarrow \\
    \leftarrow 2\tilde{\vect{w}}_1^T \rightarrow \\
    \vdots \\
    \leftarrow 2\tilde{\vect{w}}_C^T \rightarrow
\end{matrix}\right] =
\frac{1}{\sigma_w^2}W^T
\]

Substituting this all back into the gradient:

\[ \frac{d\mathcal{C}}{dW} = -Y\tilde{X}^T + P\tilde{X}^T + \frac{1}{\sigma_w^2}W^T \]
\[ \frac{d\mathcal{C}}{dW} = (P-Y)\tilde{X}^T + \frac{1}{\sigma_w^2}W^T \]

\subsubsection{Gradient descent}

Start with:
\[ W[0] = 0 \]
Iterate:
\[ W[i+1] = W[i] - \eta {\left(\frac{d\mathcal{C}}{dW}\right)}^T \]
\[ W[i+1] = W[i] - \eta\left(\tilde{X}{(P - Y)}^T + \frac{1}{\sigma_w^2}W\right)\]
Until convergence, where:
\[ {\left|\left|\frac{d\mathcal{C}}{dW}\right|\right|}_F < \epsilon \]
Where $\eta$ is the learning rate and $\epsilon$ is some threshold for the gradient norm.

${\left|\left|\frac{d\mathcal{C}}{dW}\right|\right|}_F$ is the Frobenius norm, and is defined as:
\[ ||A||_F = \sqrt{\sum_i\sum_j A_{ij}^2} \]
It is just a sum of the $L_2$ norms of the columns (or rows):
\begin{align*}
||A||_F^2 &= \sum_i\sum_j A_{ij}^2 \\
          &= \sum_j \sum_i {[\vect{a}_j]}_i^2 \\
          &= \sum_j |\vect{a}_j|^2 \\
\end{align*}

\end{document}
