\documentclass[../../main.tex]{subfiles}
\begin{document}

\subsubsection{Notation}

All vectors $\vect{x}$ are column vectors:
\[
\vect{x} = \left[\begin{matrix}
        x_1\\
        x_2\\
        \vdots\\
        x_n
\end{matrix}\right]
\]

A row vector is denoted with $\vect{x}^T = \left[ x_1 x_2 \cdots  x_n\right]$.

In general, a matrix $A$ should have shape $m\times n$, $A \in \mathbb{R}^{m\times n}$, meaning there are $m$ rows and $n$ columns. Therefore, $A$ transforms a vector $\vect{x} \in \mathbb{R}^n$ to a matrix $\vect{y} = A\vect{x} \in \mathbb{R}^m$.

Matrices are indexed by their row number, then column number, such that $A_{ij}$ is the i'th row and j'th column.

$\vect{a}_j$ refers the j'th column of a matrix.

$\hat{\vect{a}}_i$ refers the i'th row of a matrix, as a column vector, not a row vector. To express it as a row vector, write $\hat{\vect{a}}_i^T$.

ie:
\[ A =
\left[\begin{matrix}
        \uparrow & \uparrow & & \uparrow \\
        \vect{a}_1 & \vect{a}_2 & \cdots & \vect{a}_n \\
        \downarrow & \downarrow & & \downarrow
\end{matrix}\right] =
\left[\begin{matrix}
    \leftarrow {\vect{a_1}'}^T \rightarrow \\
    \leftarrow {\vect{a_2}'}^T \rightarrow \\
    \cdots \\
    \leftarrow {\vect{a_m}'}^T \rightarrow \\
\end{matrix}\right]
\]


When using a matrix $X$ to hold vectors $\vect{x}$, do so as column vectors to remain consistent.
\[
X = \left[\begin{matrix}
        \uparrow & \uparrow & & \uparrow \\
        \vect{x}_1 & \vect{x}_2 & \cdots & \vect{x}_n \\
        \downarrow & \downarrow & & \downarrow
\end{matrix}\right]
\]

\subsubsection{Matrix calculus definition}

I will use \textit{numerator-layout notation} throughout, which means that a derivative has the same form as its numerator.

\[ \frac{dy}{dx} = \textrm{scalar}\]

\[ \frac{d\vect{y}}{dx} =
\left[\begin{matrix}
    \frac{dy_1}{dx} \\
    \vdots \\
    \frac{dy_n}{dx}
\end{matrix}\right] =
\textrm{column vector}
\]

\[ \frac{dy}{d\vect{x}} =
\left[
    \frac{dy}{dx_1} \cdots \frac{dy}{dx_n}
\right] =
\textrm{row vector}
\]

\[
\textrm{But  } \nabla_{\vect{x}}(y) =
\left[\begin{matrix}
    \frac{dy}{dx_1} \\
    \vdots \\
    \frac{dy}{dx_n}
\end{matrix}\right] =
{\left(\frac{dy}{d\vect{x}}\right)}^T =
\textrm{column vector}
\]

\[
\frac{d\vect{y}}{d\vect{x}} =
\left[\begin{matrix}
    \frac{dy_1}{dx_1} & \frac{dy_1}{dy_2} & \cdots & \frac{dy_1}{dx_n}\\
    \frac{dy_2}{dx_1} & \ddots & & \\
    \vdots & &  & \\
    \frac{dy_m}{dx_1} & & & \frac{dy_m}{dx_n}
\end{matrix}\right] =
\left[\begin{matrix}
        \uparrow & \uparrow & & \uparrow \\
        \frac{d\vect{y}}{dx_1} & \frac{d\vect{y}}{x_2} & \cdots & \frac{d\vect{y}}{dx_n} \\
        \downarrow & \downarrow & & \downarrow
\end{matrix}\right] =
\left[\begin{matrix}
        \leftarrow \frac{dy_1}{d\vect{x}} \rightarrow \\
        \leftarrow \frac{dy_2}{d\vect{x}} \rightarrow \\
        \cdots \\
        \leftarrow \frac{dy_m}{d\vect{x}} \rightarrow \\
\end{matrix}\right] =
\textrm{matrix}
\]

The above is equivalent to the standard form of the Jacobian:
\[ \frac{d\vect{u}}{d\vect{v}} = \frac{\partial(u_1, \ldots, u_m)}{\partial(v_1, \ldots, v_n)} =
\left[\begin{matrix}
    \frac{du_1}{dv_1} & \frac{du_1}{dv_2} & \cdots & \frac{du_1}{dv_n}\\
    \frac{du_2}{dv_1} & \ddots & & \\
    \vdots & &  & \\
    \frac{du_m}{dv_1} & & & \frac{dy_m}{dv_n}
\end{matrix}\right]
\]

For matrix $A$ and scalar $\phi$.

\[ {\left[\frac{dA}{d\phi}\right]}_{ij} = \frac{dA_{ij}}{d\phi}
\quad \textrm{Matching indices}\]

\[ {\left[\frac{d\phi}{dA}\right]}_{ij} = \frac{d\phi}{dA_{ji}}
\quad \textrm{Opposite indices}\]

As expected, for numerator-layout notation, the matrix indices match when $A$ is the numerator, and are swapped (transpose) when $A$ is the denominator.
This is equivalent to the fact that when a (column) vector is the denominator, this produces a gradient with the form of the transpose, a row vector.

The matrix derivative is consistent with the form of the derivatives of its columns and rows.

\[ \frac{d\phi}{dA} =
\left[\begin{matrix}
    \frac{d\phi}{dA_{11}} & \frac{d\phi}{dA_{21}} & \cdots \\
    \frac{d\phi}{dA_{12}} & \ddots & \\
    \vdots & &
\end{matrix}\right] =
\left[\begin{matrix}
        \leftarrow \frac{d\phi}{d\vect{a}_1} \rightarrow \\
        \leftarrow \frac{d\phi}{d\vect{a}_2} \rightarrow \\
        \dots \\
        \leftarrow \frac{d\phi}{d\vect{a}_n} \rightarrow \\
\end{matrix}\right] =
\left[\begin{matrix}
        \uparrow & \uparrow & & \uparrow \\
    {\left(\frac{d\phi}{d\hat{\vect{a}}_1}\right)}^T &
    {\left(\frac{d\phi}{d\hat{\vect{a}}_1}\right)}^T &
    \cdots &
    {\left(\frac{d\phi}{d\hat{\vect{a}}_1}\right)}^T \\
        \downarrow & \downarrow & & \downarrow
\end{matrix}\right]
\]

ie: Differentiating by the columns of $A$ gives the rows of $\frac{d\phi}{dA}$ and differentiating by the rows of $A$ gives the columns of $\frac{d\phi}{dA}$.

The only reason a transpose is needed around ${\left(\frac{d\phi}{\hat{\vect{a}}_i}\right)}^T$ is because of the convention that all vectors are column vectors. If the rows $\hat{\vect{a}}_i$ were left as row vectors, and differentiating with respect to a row vector gives a column vector, this would naturally fit the columns of $\frac{d\phi}{dA}$.

\subsubsection{Matrix calculus identities}

\end{document}
